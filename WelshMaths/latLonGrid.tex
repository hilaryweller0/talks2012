\begin{slide}
\heading{What is so good about the lat-lon grid}\pauseHS

\begin{list0}
\item Direct addressing\pauseHS
\item High order accuracy simple and cheap\pauseHS
\item Geostrophic balance maintained\pauseHS
\item Good conservation properties\pauseHS
\item Orthogonal\pauseHS
\item Good wave dispersion can use Arakawa C-grid\pauseHS
\end{list0}

For solving wave equations (with rotation)\\
\begin{center}\begin{tabular}{ccc}
$\frac{\partial \vec{u}}{\partial t} + f \vec{k}\times\vec{u} = g\nabla h$ &
\hspace{1.5cm} &
$\frac{\partial h}{\partial t} + H \nabla\dprod\vec{u} = 0$
\end{tabular}\end{center}

\href{file:///home/hilary/latex/meetings/2012/WelshMaths/ArakawaGrids.html}
{\begin{tabular}{ccc} A-grid & B-grid & C-grid \\
\includegraphics[width=0.3\linewidth]{../../2011/climateSummerSchool/HilaryNotes/2dSWE/Agrid0.png}&
\includegraphics[width=0.3\linewidth]{../../2011/climateSummerSchool/HilaryNotes/2dSWE/Bgrid0.png}&
\includegraphics[width=0.3\linewidth]{../../2011/climateSummerSchool/HilaryNotes/2dSWE/Cgrid0.png}
\end{tabular}}

%\href{run:graphics/A.gif}{hmm}

\end{slide}

\begin{slide}
\heading{Problem with lat-lon}\pauseHS

\begin{list0}
\item Convergence of meridians towards the poles\pauseHS\\
    $\implies$ need numerical method to circumvent time-step restriction. Eg:
    \begin{list1}
    \item Semi-implicit
    \item Semi-Lagrangian
    \item Fourier filtering of short wavelengths\pauseHS
    \end{list1}
    
\item In order to take longer time-step you must increase the domain of dependence\\
    $\rightarrow$ more interprocessor communication\\
    $\rightarrow$ scaling bottlenecks\pauseHS

\item $\therefore$ we need a quasi-uniform grid
\end{list0}

\end{slide}
