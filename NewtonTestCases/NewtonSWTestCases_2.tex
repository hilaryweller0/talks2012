%% LyX 2.0.3 created this file.  For more info, see http://www.lyx.org/.
%% Do not edit unless you really know what you are doing.
\documentclass[12pt,english]{article}
\usepackage{mathptmx}
\usepackage[T1]{fontenc}
\usepackage[pdftex]{geometry}
\geometry{verbose,tmargin=2cm,bmargin=2cm,lmargin=2cm,rmargin=2cm,footskip=1cm}
\setcounter{secnumdepth}{5}
\setcounter{tocdepth}{5}
\setlength{\parskip}{\medskipamount}
\setlength{\parindent}{0pt}
\usepackage{array}
\usepackage{multirow}
\usepackage{amsmath}
\usepackage[pdftex]{graphicx}
\usepackage[authoryear]{natbib}

\makeatletter

%%%%%%%%%%%%%%%%%%%%%%%%%%%%%% LyX specific LaTeX commands.
%% Because html converters don't know tabularnewline
\providecommand{\tabularnewline}{\\}

%%%%%%%%%%%%%%%%%%%%%%%%%%%%%% User specified LaTeX commands.
% Uncomment to print out only slides and overlays
%
%\onlyslides{\slides}

% Uncomment to print out only notes
%
%\onlynotes{\notes}


\newcommand{\nicefrac}[2]{\ensuremath ^{#1}\!\!/\!_{#2}}
\usepackage { fancybox}
\usepackage{tweaklist}
\renewcommand{\itemhook}
{
    \setlength{\topsep}{2pt}
    \setlength{\parsep}{2pt}
    \setlength{\itemsep}{2pt}
    \setlength{\labelwidth}{1ex}
    \setlength{\leftmargin}{3ex}
    \raggedright
}
\usepackage[pdftex]{graphicx,color}

\makeatother

\usepackage{babel}
\begin{document}

\title{Shallow Water Test Cases with Static Refinement}


\author{Hilary Weller <h.weller@reading.ac.uk>}


\date{4/10/12}

\maketitle

\section{Definition of Resolution and Refinement\label{sec:rhoDefn}}
\begin{itemize}
\item The mesh density is defined as $\rho=\frac{\text{dof}}{3A}$ where
dof is the total number of degrees of freedom of the shallow water
model over a region with area $A$. The factor of 3 is because dof
includes all velocity and height degrees of freedom, so dividing dof
by $3A$ to give $\rho$ gives approximate grid points per unit area.
Therefore $1/\sqrt{\rho}$ is the equivalent mesh spacing.
\item A region refined a factor of $k$ will have $\rho_{r}=k^{2}\rho_{c}$
where $\rho_{c}$ is the mesh density in the coarse region
\end{itemize}

\section{Test Case on a Plane}

Non-linear shallow water equations on a $\beta$ plane: (advective
form but model dependent)
\begin{eqnarray*}
\frac{\partial u}{\partial t}+\mathbf{u}\cdot\nabla u & = & \phantom{+}fv-g\frac{\partial h}{\partial x}\\
\frac{\partial v}{\partial t}+\mathbf{u}\cdot\nabla v & = & -fu-g\frac{\partial h}{\partial y}\\
\frac{\partial h}{\partial t}+\mathbf{u}\cdot\nabla h & + & h\nabla\cdot\mathbf{u}=0
\end{eqnarray*}
where $u$ and $v$ and the velocities in the $x$ and $y$ directions,
$\mathbf{u}=\left(\begin{array}{c}
u\\
v
\end{array}\right)$, $h$ is the shallow water height and $f=\beta y$ is the Coriolis
and $g$ is the acceleration due to gravity.


\subsection*{Analytic solution: a geostrophically balanced, compact jet}

\begin{eqnarray*}
u(\hat{y}) & = & \begin{cases}
u_{0}\left(1-3\hat{y}^{2}+3\hat{y}^{4}-\hat{y}^{6}\right) & \text{for }-1\le\hat{y}\le1\\
0 & \text{otherwise}
\end{cases}\\
v & = & 0\\
h\left(\hat{y}\right) & = & \begin{cases}
h_{0} & \textnormal{for }\hat{y}\le-1\\
h_{0}-\frac{w\beta u_{0}}{g}\begin{array}[t]{l}
\biggl\{ y_{c}\left(\frac{16}{35}+\hat{y}-\hat{y}^{3}+\frac{3}{5}\hat{y}^{5}-\frac{1}{7}\hat{y}^{7}\right)\\
+w\left(-\frac{1}{8}+\frac{1}{2}\hat{y}^{2}-\frac{3}{4}\hat{y}^{4}+\frac{1}{2}\hat{y}^{6}-\frac{1}{8}\hat{y}^{8}\right)\biggr\}
\end{array} & \textnormal{for }-1\le\hat{y}\le1\\
h_{0}-\frac{32}{35}\frac{w\beta u_{0}y_{c}}{g} & \textnormal{for }\hat{y}\ge1
\end{cases}
\end{eqnarray*}
where $\hat{y}=\frac{y-y_{c}}{w}$, $y_{c}$ is the $y$ position
of the jet centre, $w$ is the jet half width, $u_{0}$ is the maximum
jet velocity and $h_{0}$ is the minimum height. The jet velocity
and height as a function of $y$ using the parameter values in the
table below are shown in figure 

\begin{figure}
\noindent \begin{centering}
\includegraphics[width=0.75\textwidth]{figs/jetProfile}
\par\end{centering}

\caption{Profile of height, $h$, and velocity, $u$, in the geostrophically
balanced jet}


\end{figure}


An analytic solution of the shallow water equations on a $\beta$
plane consisting of a compact, zonal jet with zero first and second
derivatives at the edges is given by:


\subsection{Mesh and Problem specifications}

\begin{tabular}{lc}
$\hat{y}=\frac{y-y_{c}}{w}$ & \multirow{8}{*}{\input{figs/jet.pdf_t}}\tabularnewline
$a=6\times10^{6}\text{m}$ & \tabularnewline
$u_{0}=20\text{ms}^{-1}$ & \tabularnewline
$y_{c}=a$ & \tabularnewline
$w=\frac{a}{2}$ & \tabularnewline
$g=10\text{ms}^{-2}$ & \tabularnewline
$\beta=2\times10^{-11}\text{s}^{-1}\text{m}^{-1}$ & \tabularnewline
$h_{0}=3000\text{m}$ & \tabularnewline
\end{tabular}\vspace{6cm}

\begin{itemize}
\item The geometry is defined as in the above figure and goes from $x=0\rightarrow2a$
and $y=0\rightarrow2a$. 
\item The resolution must conform to two specifications, $\rho_{c}$, the
mesh density in the coarse region and $\rho_{r}$, the mesh density
in the refined region. The coarse region must include at least the
region between $x=a/4$ and $x=3a/4$ for all $y$. The refined region
must include at least the circle centered at $\left(3a/2,\ 3a/4\right)$with
radius $a/4$.
\item The refined region can be larger than this to allow for your refinement
technique (ie blocks, unstructured, smooth etc) but the smoother you
make it the more unnecessary degrees of freedom you will add.
\item There should be four meshes for every resolution, with $\rho_{r}=k^{2}\rho_{c}$
with $k=1,2,4$ and $8$.
\item The standard resolution is $\rho_{c}=\frac{100\times100}{4a^{2}}$
where $4a^{2}$ is the area of the domain.
\item For convergence with resolution, value of $\rho_{c}$ of $\frac{24\times24}{4a^{2}}$,
$\frac{50\times50}{4a^{2}}$, and $\frac{200\times200}{4a^{2}}$ should
also be used.
\item Cyclic in x direction
\item No normal flow (slip walls) in y direction, ie $v=0$ $\implies\frac{\partial h}{\partial y}=-\frac{fu}{g}$
at $y=0,\ 2a$
\item The test case can be run on a uniform, aligned, orthogonal grid. This
will give useful information about the impact of refinement. This
should be compared with a grid designed to reflect the need to grid
a sphere. Eg:

\begin{itemize}
\item non-orthogonal blocks with angles of up to $120^{o}$ (this can be
done by moving the edge of a grid line up to $\frac{a}{2}\left(1+\frac{1}{2\sqrt{3}}\right)$
and the centre of the same grid line down to $\frac{a}{2}\left(1-\frac{1}{2\sqrt{3}}\right)$.) 
\item a mix of hexagons and pentagons and possibly other shapes
\item triangles with 5 or 6 vertex neighbours
\end{itemize}
\item The simulation is run for $6\times10^{5}s$ (one complete revolution)
\end{itemize}
The initial conditions on the grid with $\rho_{c}=\frac{50\times50}{4a^{2}}$
with $k=2$ are shown below (only every 25th vector is shown)

\includegraphics[scale=0.7]{graphics/cubeSphere_r_50x50_0_hU}


\section{Test Cases on the Sphere}
\begin{itemize}
\item \citet{WDH+92} test cases 2, 3 and 5 with and without a statically
refined region
\item The steady state test cases should be run for 12 days. The mountain
test case should be run for 15 days
\item The minimum size of the statically refined region is the mountain
area from test case 5. 
\item The mountain must be centered on a special point of the mesh (ie a
cube corner, a pole of a rotated lat-log grid or a pentagon). This
is probably not important, except for a lat-lon grid.
\item The coarse region must cover at least the antipodal hemisphere to
the mountain.
\item The standard coarse resolution should be close to $\rho_{c}=\frac{10^{4}}{3\times4\pi a^{2}}$
where $a$ is the radius of the earth (ie $10^{4}/3$ cells for a
finite volume mesh with $k=1$, see definition of mesh density in
section \ref{sec:rhoDefn})
\item There should be four meshes for every resolution, with $\rho_{r}=k^{2}\rho_{c}$
with $k=1,2,4$ and $8$.
\item For convergence with resolution, values of $\rho_{c}$ of around $\frac{2\times10^{3}}{3\times4\pi a^{2}}$,
$\frac{4\times10^{4}}{3\times4\pi a^{2}}$ and $\frac{2\times10^{5}}{3\times4\pi a^{2}}$
should also be used.
\end{itemize}

\section{Diagnostics }
\begin{itemize}
\item Height, velocity and vorticity differences from the initial conditions
(errors) should be plotted for the standard resolution for all values
of $k$ at the end of the simulation. 
\item For the standard resolutions for all values of $k$, the following
global diagnostics should be graphed as a function of time for all
time steps:

\begin{itemize}
\item Absolute values of normalised mass, energy, enstrophy and potential
vorticity \citep[as defined by][]{WDH+92} with a log y-scale with
negative values dashed
\item For the steady state test cases, error norms $\ell_{2}(h)$, $\ell_{\infty}(h)$,
$\ell_{2}(u)$ and $\ell_{\infty}(u)$ as defined by \citet{WDH+92}
with a log y-scale
\end{itemize}
\item Graphs of $\ell_{2}(h)$, $\ell_{\infty}(h)$, $\ell_{2}(u)$ and
$\ell_{\infty}(u)$ convergence with resolution. The graphs should
be log-log and include lines showing typical convergence rates. Two
forms should be presented, one with $1000/\sqrt{\rho_{c}}$ (in km)
on the x-axis and the other with the total number of dofs on the x-axis.
(The form with $1000/\sqrt{\rho_{c}}$ on the x-axis shows the affect
of refinement on the accuracy, the form with dofs on the x-axis allows
for comparison between methods with different mesh refinement strategies.)
\end{itemize}

\section{Summary of the Numerical Method}

The salient points of the numerical method should be listed with references
to more detailed descriptions. The summary should include the following:
\begin{itemize}
\item Form of the equations solved (ie flux, advective or vector invariant
form)
\item Discretisation type (eg finite element, discontinuous Galerkin, finite
volume, spectral etc)
\item Prognostic variables used and their placement on the mesh
\item Number of degrees of freedom per element or stencil sizes for the
operators
\item Order of accuracy of the approximations for all of the spatial operators
and special techniques for each operator
\item Any limiters used
\item Time-stepping schemes for each of the terms
\item Time steps used for each term of the equations
\item Maximum theoretical time-steps
\item The size, number and condition number of any matrix solutions (eg
mass matrices or linearised Helmholtz equations)
\item The linear equation solvers and pre-conditioners
\item Number of calls to the solvers per time step
\item Typical number of iterations per solver call
\end{itemize}
\bibliographystyle{abbrvnat}
\bibliography{numerics}

\end{document}
